% !TEX root = ../../teaching_online.tex
\newpage
\section{Twitter}

% What is Twitter
\subsection{What is Twitter}

Twitter is an online social media platform described as a `microblogging' system. Users post short messages (currently 280 characters or less), photos, videos, or links. These posts are called tweets. Other uses can comment or `retweet' other uses posts as well as `follow' other users (meaning the users tweets can appear in their own personal feed of content they follow). Content can be `hash-tagged' (labeled with \#label, where the label can be anything), e.g. \#oscars, \#mondaymotivation, \#icantevenwithtoday, etc. Hashtags which are being used by many users on the platform are said to be `trending'. 



% Why Twitter?
\subsection{Why Twitter?}

Twitter is currently one of the most used social media platforms. It is easy to use and can be used to connect students with each other, you the instructor, and possible course content. For instance, you can create a Twitter account, e.g. SUmat295, SUIntroAnthropology, etc., and have students follow the account. You can then 'retweet' content relevant to the course, e.g. news articles, videos, quotes, etc., and create your own posts and have students find it using unique hashtags, e.g. \#SUmat295course. Students can then comment on your posts to give opinions. Students could then also coordinate study groups using pre-determined hashtags like \#sumat295studygroup, etc. 

This gives students a chance to interact with the course `in real time' (as can you see student interaction `in real time' from your phone), interact with material, and coordinate/communicate with other students on a platform many of them will use daily. You could include comments, posting/reposting, etc. as part of the participation in the course. [Students need not use their own personal account but rather an account they create for the course.] How you choose to use it depends on the course, the course design, and your vision, but there are many possibilities! 



% Signing Up
\subsection{Signing Up for Twitter}

Signing up for Twitter is simple. Go to \url{https://twitter.com/i/flow/signup} and enter the required information, such as your name and email address. Once you have set up an account, you need to select a unique username. You will want to choose a name which identifies the course and which is easy to remember for students but is unique enough to be `limited' to course users. After you create your account, you can complete your `bio', i.e. add a course description, and add a profile photo/header for the account. For more on this, you can watch this video: \url{https://www.youtube.com/watch?v=EUl9asmTsJw}. 



% Tweeting
\subsection{Tweeting \& Hashtags}

Once you have an account, you can begin posting/retweeting content relevant to the course. What material you are tweeting, hashtags you are using, etc depend on the course and how you are using it. However, a few of the videos below will help you get started:

	\begin{itemize}
	\item The Ultimate Guide to Twitter: \url{https://www.theedublogger.com/twitter/}
	\item How to Use Twitter (A Beginners Guide 2020)---YouTube: \url{https://www.youtube.com/watch?v=E2_em-1gCp4}
	\item How to Use Twitter---YouTube: \url{https://www.youtube.com/watch?v=5jWNpLvdocU}
	\item How Hashtags Work on Social Media---YouTube: \url{https://www.youtube.com/watch?v=rX5MbZ48EwM}
	\item How to Use Twitter: Critical Tips for New Users: \url{https://www.wired.com/story/how-to-setup-twitter-search-hashtag-and-login-help/}
	\end{itemize}



% Concerns
\subsection{Twitter Concerns}

As with using any social media in teaching, there are many things to be aware of. It is best to introduce these methods slowly in your course---a little at a time---to learn what works and does not work, as well as how to successfully engage students with the application. One should read thoroughly about teaching pedagogy using the application, experiment with the app independently, and plan thoroughly how to manage how students will use/interact with the app. As a few examples of things to think about, plan, or be concerned with:

	\begin{itemize}
	\item Privacy: Who will be able to access the course profile page? What could happen if people outside the course are able to access posts or course material? Is this a feature you want to use in the course or something you want restricted? How will links, posts, hashtags, etc by your students in the course `expose' then? What type of information does the app gather and keep track of on the students end, e.g. geographic location, internet history, etc. 
	
	\item Sensitive Materials: What types of posts, videos, and photos could the students see on the app? How will they engage others (non-students) using the app? How will you restrict (or not restrict) `objectionable' material from being shared, tagged, or posted to the course profile? What is `objectionable' (see below)? What level can students disengage if they feel uncomfortable? How are different students more or less vulnerable on the app due to their race, color, age, sex, gender (identity), sexual orientation, veteran status, economic status, appearance, disability, citizen status, national origin, religion, marital status, prior criminal status, etc.? 
	
	\item User Guidelines: Students could easily begin teaching the course profile/material as social media first and course material second. Beyond the university guidelines on discussion and ethical behavior, what rules/guidelines will you use with students on the app? What level of control will you give students over tagging, sharing, posting, discussing? What is `okay' and what is `not okay'? What if students violate these rules? What are the consequences or leniency for students? 
	\end{itemize}

